\documentclass[12pt,a4paper]{article}
\usepackage[UTF8]{ctex}
\usepackage{geometry}
\usepackage{graphicx}
\usepackage{booktabs}
\usepackage{listings}
\usepackage{xcolor}
\usepackage{hyperref}
\usepackage{enumitem}
\usepackage{fontspec}
\usepackage{titlesec}
\usepackage{fancyhdr}
\usepackage{float}
\usepackage{longtable}
\usepackage{multirow}
\usepackage{array}
\usepackage{tcolorbox}

% 页面设置
\geometry{left=2.5cm,right=2.5cm,top=2.5cm,bottom=2.5cm}

% 代码样式
\lstset{
    basicstyle=\ttfamily\small,
    keywordstyle=\color{blue},
    commentstyle=\color{gray},
    stringstyle=\color{red},
    backgroundcolor=\color{gray!10},
    frame=single,
    breaklines=true,
    showstringspaces=false,
    numbers=left,
    numberstyle=\tiny\color{gray},
    language=bash
}

% 页眉页脚
\pagestyle{fancy}
\fancyhf{}
\fancyhead[L]{Git与GitHub使用教程}
\fancyhead[R]{智能图书管理系统}
\fancyfoot[C]{\thepage}

% 标题格式
\titleformat{\section}{\Large\bfseries}{\thesection}{1em}{}
\titleformat{\subsection}{\large\bfseries}{\thesubsection}{1em}{}
\titleformat{\subsubsection}{\normalsize\bfseries}{\thesubsubsection}{1em}{}

% 提示框样式
\tcbuselibrary{skins,breakable}
\newtcolorbox{tipbox}[1][]{
    colback=blue!5,
    colframe=blue!75!black,
    fonttitle=\bfseries,
    title=提示,
    #1
}
\newtcolorbox{warnbox}[1][]{
    colback=yellow!10,
    colframe=orange!75!black,
    fonttitle=\bfseries,
    title=注意,
    #1
}
\newtcolorbox{dangerbox}[1][]{
    colback=red!5,
    colframe=red!75!black,
    fonttitle=\bfseries,
    title=警告,
    #1
}

\begin{document}

% ============================================================
% 封面
% ============================================================
\begin{titlepage}
    \centering
    \vspace*{2cm}

    {\Huge\bfseries Git与GitHub完全指南\\[0.8cm]}
    {\LARGE\bfseries 从入门到进阶\\[2cm]}

    \vfill

    {\Large
    \begin{tabular}{rl}
        \textbf{项目实例:} & 智能图书管理系统 \\[0.4cm]
        \textbf{仓库地址:} & https://github.com/wkylnj/Library-System \\[0.4cm]
        \textbf{作者:} & 王楷宇 \\[0.4cm]
    \end{tabular}
    }

    \vfill
    {\large \today}
\end{titlepage}

% 目录
\tableofcontents
\newpage

% ============================================================
\section{Git基础概念}
% ============================================================

\subsection{什么是Git}

Git是一个分布式版本控制系统,由Linux之父Linus Torvalds于2005年创建。它可以:

\begin{itemize}[leftmargin=2em]
    \item 记录文件的每次修改历史
    \item 支持多人协作开发
    \item 在不同版本之间自由切换
    \item 创建分支进行并行开发
    \item 合并不同分支的修改
\end{itemize}

\subsection{Git工作流程}

Git有三个主要区域:

\begin{enumerate}
    \item \textbf{工作区(Working Directory)}:本地文件系统中的项目目录
    \item \textbf{暂存区(Staging Area)}:准备提交的文件变更
    \item \textbf{版本库(Repository)}:已提交的历史记录
\end{enumerate}

\begin{lstlisting}[caption=Git工作流程]
工作区 ---> git add ---> 暂存区 ---> git commit ---> 版本库
                                                         |
                                                         v
                                               git push ---> 远程仓库
\end{lstlisting}

\subsection{GitHub简介}

GitHub是全球最大的代码托管平台,提供:

\begin{itemize}[leftmargin=2em]
    \item 远程Git仓库托管
    \item 代码协作与审查(Pull Request)
    \item 项目管理(Issues、Projects)
    \item 自动化工作流(GitHub Actions)
    \item 开源社区交流
\end{itemize}

% ============================================================
\section{SSH密钥配置}
% ============================================================

使用SSH协议连接GitHub可以避免每次操作都输入用户名和密码。

\subsection{检查现有SSH密钥}

\begin{lstlisting}[caption=检查SSH密钥]
# 查看是否已有SSH密钥
ls -la ~/.ssh

# 如果存在 id_rsa.pub 或 id_ed25519.pub,则已有密钥
\end{lstlisting}

\subsection{生成新的SSH密钥}

\begin{lstlisting}[caption=生成SSH密钥]
# 使用Ed25519算法(推荐)
ssh-keygen -t ed25519 -C "your_email@example.com"

# 或使用RSA算法(兼容性更好)
ssh-keygen -t rsa -b 4096 -C "your_email@example.com"
\end{lstlisting}

执行后会提示:

\begin{lstlisting}
Enter file in which to save the key (/root/.ssh/id_ed25519):
# 直接回车使用默认路径

Enter passphrase (empty for no passphrase):
# 可选:设置密码短语,增加安全性

Enter same passphrase again:
# 确认密码短语
\end{lstlisting}

\subsection{添加SSH密钥到ssh-agent}

\begin{lstlisting}[caption=启动ssh-agent并添加密钥]
# 启动ssh-agent
eval "$(ssh-agent -s)"

# 添加私钥到ssh-agent
ssh-add ~/.ssh/id_ed25519
# 或
ssh-add ~/.ssh/id_rsa
\end{lstlisting}

\subsection{将公钥添加到GitHub}

\begin{lstlisting}[caption=复制公钥内容]
# 查看并复制公钥
cat ~/.ssh/id_ed25519.pub
# 或
cat ~/.ssh/id_rsa.pub
\end{lstlisting}

然后在GitHub上添加:

\begin{enumerate}
    \item 登录GitHub,点击右上角头像
    \item 选择 Settings $\rightarrow$ SSH and GPG keys
    \item 点击 "New SSH key"
    \item Title填写描述(如:My Server)
    \item Key粘贴刚才复制的公钥内容
    \item 点击 "Add SSH key"
\end{enumerate}

\subsection{测试SSH连接}

\begin{lstlisting}[caption=测试GitHub SSH连接]
ssh -T git@github.com

# 成功会显示:
# Hi username! You've successfully authenticated, but GitHub
# does not provide shell access.
\end{lstlisting}

\begin{tipbox}
首次连接会提示确认GitHub的指纹,输入 \texttt{yes} 即可。
\end{tipbox}

% ============================================================
\section{将新项目提交到GitHub}
% ============================================================

以智能图书管理系统为例,演示如何将一个新项目提交到GitHub。

\subsection{在GitHub创建仓库}

\begin{enumerate}
    \item 登录GitHub
    \item 点击右上角 "+" $\rightarrow$ "New repository"
    \item 填写仓库名称:Library-System
    \item 选择公开(Public)或私有(Private)
    \item 不要勾选 "Initialize this repository with a README"(如果本地已有项目)
    \item 点击 "Create repository"
\end{enumerate}

\subsection{初始化本地Git仓库}

\begin{lstlisting}[caption=初始化Git仓库]
# 进入项目目录
cd /root/website_homework/library_system

# 初始化Git仓库
git init

# 查看状态
git status
\end{lstlisting}

\subsection{创建.gitignore文件}

\begin{lstlisting}[caption=.gitignore文件内容]
# Python
__pycache__/
*.py[cod]
*.so
.Python
*.egg-info/
venv/
env/

# Django
*.log
db.sqlite3
*.sqlite3

# IDE
.idea/
.vscode/

# 敏感文件
.env
*.pem
cookies.txt

# OS
.DS_Store
Thumbs.db
\end{lstlisting}

\begin{warnbox}
务必在提交前创建.gitignore,避免将敏感信息(如API密钥、数据库)提交到仓库!
\end{warnbox}

\subsection{处理敏感信息}

将敏感配置改为环境变量:

\begin{lstlisting}[language=Python,caption=settings.py中的敏感信息处理]
import os

# 错误示例 - 不要这样做!
# AI_API_KEY = 'sk-your-secret-key'

# 正确示例 - 使用环境变量
AI_API_KEY = os.environ.get('AI_API_KEY', '')
EMAIL_HOST_PASSWORD = os.environ.get('EMAIL_HOST_PASSWORD', '')
\end{lstlisting}

创建.env.example文件供参考:

\begin{lstlisting}[caption=.env.example]
# AI服务配置
AI_API_KEY=your_api_key_here

# 邮件配置
EMAIL_HOST_PASSWORD=your_password_here
\end{lstlisting}

\subsection{添加远程仓库}

\begin{lstlisting}[caption=添加远程仓库]
# 添加远程仓库(使用SSH)
git remote add origin git@github.com:wkylnj/Library-System.git

# 查看远程仓库
git remote -v

# 输出:
# origin  git@github.com:wkylnj/Library-System.git (fetch)
# origin  git@github.com:wkylnj/Library-System.git (push)
\end{lstlisting}

\subsection{提交并推送}

\begin{lstlisting}[caption=首次提交]
# 重命名默认分支为main(现代规范)
git branch -m main

# 添加所有文件到暂存区
git add -A

# 查看将要提交的文件
git status

# 提交
git commit -m "Initial commit: 智能图书管理系统

功能特性:
- AI智能对话助手
- AI个性化图书推荐
- AI图书封面生成
- 图书副本管理
- 预约与到期提醒系统"

# 推送到远程仓库
git push -u origin main
\end{lstlisting}

\begin{tipbox}
\texttt{-u} 参数会设置上游分支,之后可以直接使用 \texttt{git push} 和 \texttt{git pull}。
\end{tipbox}

% ============================================================
\section{远程仓库已有内容的情况}
% ============================================================

当远程仓库已有内容时,直接推送会被拒绝。有以下几种处理方式:

\subsection{场景一:合并远程内容}

当你想保留远程仓库的历史记录时:

\begin{lstlisting}[caption=拉取并合并]
# 拉取远程内容(允许不相关的历史)
git pull origin main --allow-unrelated-histories

# 如果有冲突,手动解决后:
git add .
git commit -m "Merge remote changes"

# 推送
git push origin main
\end{lstlisting}

\subsection{场景二:强制覆盖远程}

当你确定要用本地版本完全替换远程版本时:

\begin{lstlisting}[caption=强制推送]
# 强制推送(会覆盖远程所有内容!)
git push -f origin main
\end{lstlisting}

\begin{dangerbox}
强制推送会丢失远程仓库的所有历史记录!请确保:
\begin{itemize}
    \item 远程仓库的内容确实不需要保留
    \item 没有其他协作者依赖远程仓库
    \item 已经备份了重要内容
\end{itemize}
\end{dangerbox}

\subsection{场景三:下载源码修改后提交}

如果你是直接下载(而非clone)了别人的源码并修改:

\begin{lstlisting}[caption=处理下载的源码]
# 1. 在项目目录初始化git
cd your-project
git init

# 2. 添加远程仓库
git remote add origin git@github.com:username/repo.git

# 3. 拉取远程内容
git fetch origin

# 4. 创建本地分支并跟踪远程
git checkout -b main origin/main

# 5. 此时会提示有冲突,因为本地文件与远程不同
# 如果要保留你的修改,可以:
git checkout --ours .  # 保留本地版本
# 或
git checkout --theirs .  # 保留远程版本

# 6. 手动合并需要的修改后提交
git add .
git commit -m "Add my modifications"
git push origin main
\end{lstlisting}

\subsection{场景四:本地版本落后于远程}

当其他人推送了新的提交,你的本地版本落后时:

\begin{lstlisting}[caption=更新本地版本]
# 方法1:拉取并合并
git pull origin main

# 方法2:拉取并变基(保持线性历史)
git pull --rebase origin main

# 方法3:先获取再决定如何合并
git fetch origin
git log origin/main  # 查看远程新提交
git merge origin/main  # 合并
# 或
git rebase origin/main  # 变基
\end{lstlisting}

% ============================================================
\section{Git基础操作}
% ============================================================

\subsection{配置用户信息}

\begin{lstlisting}[caption=配置Git用户信息]
# 配置用户名
git config --global user.name "Your Name"

# 配置邮箱
git config --global user.email "your_email@example.com"

# 查看配置
git config --list
\end{lstlisting}

\subsection{克隆仓库}

\begin{lstlisting}[caption=克隆仓库]
# 使用SSH克隆(推荐)
git clone git@github.com:wkylnj/Library-System.git

# 使用HTTPS克隆
git clone https://github.com/wkylnj/Library-System.git

# 克隆到指定目录
git clone git@github.com:wkylnj/Library-System.git my-project

# 克隆指定分支
git clone -b develop git@github.com:wkylnj/Library-System.git
\end{lstlisting}

\subsection{查看状态与历史}

\begin{lstlisting}[caption=查看状态]
# 查看工作区状态
git status

# 简洁模式
git status -s

# 查看提交历史
git log

# 简洁历史(单行)
git log --oneline

# 图形化显示分支
git log --oneline --graph --all

# 查看最近5次提交
git log -5
\end{lstlisting}

\subsection{添加与提交}

\begin{lstlisting}[caption=添加与提交]
# 添加单个文件
git add filename.py

# 添加多个文件
git add file1.py file2.py

# 添加所有修改的文件
git add .
git add -A

# 添加所有.py文件
git add *.py

# 提交
git commit -m "提交信息"

# 添加并提交(只对已跟踪的文件有效)
git commit -am "提交信息"
\end{lstlisting}

\subsection{查看差异}

\begin{lstlisting}[caption=查看差异]
# 查看工作区与暂存区的差异
git diff

# 查看暂存区与最新提交的差异
git diff --staged

# 查看两个提交之间的差异
git diff commit1 commit2

# 查看某个文件的修改
git diff filename.py
\end{lstlisting}

\subsection{撤销操作}

\begin{lstlisting}[caption=撤销操作]
# 撤销工作区的修改(恢复到暂存区状态)
git checkout -- filename.py

# 撤销暂存(从暂存区移除,保留工作区修改)
git reset HEAD filename.py

# 撤销最近一次提交(保留修改在工作区)
git reset --soft HEAD^

# 撤销最近一次提交(保留修改在暂存区)
git reset --mixed HEAD^

# 撤销最近一次提交(彻底删除修改)
git reset --hard HEAD^

# 创建一个新提交来撤销指定提交
git revert commit_hash
\end{lstlisting}

\begin{warnbox}
\texttt{git reset --hard} 会丢失所有未提交的修改,请谨慎使用!
\end{warnbox}

% ============================================================
\section{分支操作}
% ============================================================

\subsection{分支基础}

\begin{lstlisting}[caption=分支操作]
# 查看所有分支
git branch

# 查看远程分支
git branch -r

# 查看所有分支(本地+远程)
git branch -a

# 创建新分支
git branch feature-login

# 切换分支
git checkout feature-login

# 创建并切换分支
git checkout -b feature-login

# 删除分支
git branch -d feature-login

# 强制删除未合并的分支
git branch -D feature-login

# 重命名分支
git branch -m old-name new-name
\end{lstlisting}

\subsection{合并分支}

\begin{lstlisting}[caption=合并分支]
# 切换到目标分支
git checkout main

# 合并feature分支到main
git merge feature-login

# 合并时创建合并提交(即使可以快进)
git merge --no-ff feature-login

# 取消正在进行的合并
git merge --abort
\end{lstlisting}

\subsection{变基(Rebase)}

\begin{lstlisting}[caption=变基操作]
# 将当前分支变基到main
git rebase main

# 交互式变基(可以修改、合并、删除提交)
git rebase -i HEAD~3

# 取消变基
git rebase --abort

# 继续变基(解决冲突后)
git rebase --continue
\end{lstlisting}

\begin{tipbox}
变基可以保持更干净的提交历史,但不要对已推送的提交进行变基!
\end{tipbox}

% ============================================================
\section{远程操作}
% ============================================================

\subsection{远程仓库管理}

\begin{lstlisting}[caption=远程仓库管理]
# 查看远程仓库
git remote -v

# 添加远程仓库
git remote add origin git@github.com:user/repo.git

# 修改远程仓库URL
git remote set-url origin git@github.com:user/new-repo.git

# 删除远程仓库
git remote remove origin

# 重命名远程仓库
git remote rename origin upstream
\end{lstlisting}

\subsection{推送与拉取}

\begin{lstlisting}[caption=推送与拉取]
# 推送到远程
git push origin main

# 推送并设置上游分支
git push -u origin main

# 推送所有分支
git push --all origin

# 推送标签
git push origin v1.0.0
git push --tags

# 强制推送(危险!)
git push -f origin main

# 拉取远程更新
git pull origin main

# 拉取并变基
git pull --rebase origin main

# 获取远程更新(不合并)
git fetch origin
\end{lstlisting}

\subsection{远程分支操作}

\begin{lstlisting}[caption=远程分支操作]
# 推送本地分支到远程
git push origin feature-login

# 删除远程分支
git push origin --delete feature-login

# 创建本地分支跟踪远程分支
git checkout -b feature origin/feature

# 查看本地分支与远程分支的关联
git branch -vv
\end{lstlisting}

% ============================================================
\section{Git进阶操作}
% ============================================================

\subsection{暂存工作(Stash)}

当你需要临时切换分支但不想提交当前修改时:

\begin{lstlisting}[caption=暂存工作]
# 暂存当前修改
git stash

# 暂存并添加描述
git stash save "正在开发登录功能"

# 查看暂存列表
git stash list

# 恢复最近的暂存
git stash pop

# 恢复指定的暂存
git stash apply stash@{0}

# 删除暂存
git stash drop stash@{0}

# 清空所有暂存
git stash clear
\end{lstlisting}

\subsection{标签管理}

\begin{lstlisting}[caption=标签管理]
# 创建轻量标签
git tag v1.0.0

# 创建附注标签(推荐)
git tag -a v1.0.0 -m "版本1.0.0发布"

# 查看所有标签
git tag

# 查看标签详情
git show v1.0.0

# 给历史提交打标签
git tag -a v0.9.0 commit_hash

# 推送标签到远程
git push origin v1.0.0
git push origin --tags

# 删除本地标签
git tag -d v1.0.0

# 删除远程标签
git push origin --delete v1.0.0
\end{lstlisting}

\subsection{Cherry-pick}

选择性地将某个提交应用到当前分支:

\begin{lstlisting}[caption=Cherry-pick]
# 应用单个提交
git cherry-pick commit_hash

# 应用多个提交
git cherry-pick commit1 commit2 commit3

# 应用提交但不自动提交
git cherry-pick -n commit_hash

# 取消cherry-pick
git cherry-pick --abort
\end{lstlisting}

\subsection{修改历史提交}

\begin{lstlisting}[caption=修改历史提交]
# 修改最近一次提交信息
git commit --amend -m "新的提交信息"

# 修改最近一次提交(添加遗漏的文件)
git add forgotten_file.py
git commit --amend --no-edit

# 交互式修改最近3次提交
git rebase -i HEAD~3
# 在编辑器中:
# pick -> edit: 修改提交
# pick -> squash: 合并到上一个提交
# pick -> drop: 删除提交
\end{lstlisting}

\begin{dangerbox}
修改已推送的提交历史后需要强制推送,这会影响其他协作者!
\end{dangerbox}

\subsection{查找问题提交(Bisect)}

当你知道某个版本有bug,但不确定是哪个提交引入的:

\begin{lstlisting}[caption=二分查找问题提交]
# 开始二分查找
git bisect start

# 标记当前版本有问题
git bisect bad

# 标记已知的正常版本
git bisect good v1.0.0

# Git会自动checkout中间的提交
# 测试后标记
git bisect good  # 如果没问题
git bisect bad   # 如果有问题

# 找到问题提交后结束
git bisect reset
\end{lstlisting}

\subsection{子模块(Submodule)}

在项目中引用其他Git仓库:

\begin{lstlisting}[caption=子模块操作]
# 添加子模块
git submodule add git@github.com:user/library.git libs/library

# 克隆包含子模块的项目
git clone --recursive git@github.com:user/project.git

# 更新子模块
git submodule update --init --recursive

# 更新子模块到最新
git submodule update --remote
\end{lstlisting}

\subsection{清理仓库}

\begin{lstlisting}[caption=清理仓库]
# 删除未跟踪的文件(预览)
git clean -n

# 删除未跟踪的文件
git clean -f

# 删除未跟踪的文件和目录
git clean -fd

# 删除忽略的文件
git clean -fX

# 垃圾回收
git gc

# 压缩仓库
git gc --aggressive
\end{lstlisting}

% ============================================================
\section{解决冲突}
% ============================================================

\subsection{冲突产生的原因}

当两个分支修改了同一文件的同一位置时,Git无法自动合并,需要手动解决。

\subsection{冲突标记}

\begin{lstlisting}[caption=冲突标记示例]
<<<<<<< HEAD
这是当前分支的内容
=======
这是要合并的分支的内容
>>>>>>> feature-branch
\end{lstlisting}

\subsection{解决冲突步骤}

\begin{lstlisting}[caption=解决冲突]
# 1. 查看冲突文件
git status

# 2. 编辑冲突文件,删除冲突标记,保留需要的内容

# 3. 标记冲突已解决
git add conflicted_file.py

# 4. 完成合并
git commit -m "Resolve merge conflict"
\end{lstlisting}

\subsection{使用工具解决冲突}

\begin{lstlisting}[caption=使用合并工具]
# 使用默认合并工具
git mergetool

# 指定合并工具
git mergetool --tool=vimdiff
\end{lstlisting}

% ============================================================
\section{最佳实践}
% ============================================================

\subsection{提交规范}

\begin{lstlisting}[caption=提交信息格式]
<type>(<scope>): <subject>

<body>

<footer>
\end{lstlisting}

常用type:
\begin{itemize}[leftmargin=2em]
    \item \textbf{feat}: 新功能
    \item \textbf{fix}: 修复bug
    \item \textbf{docs}: 文档更新
    \item \textbf{style}: 代码格式调整
    \item \textbf{refactor}: 重构
    \item \textbf{test}: 测试相关
    \item \textbf{chore}: 构建/工具相关
\end{itemize}

示例:
\begin{lstlisting}
feat(auth): 添加用户登录功能

- 实现邮箱密码登录
- 添加登录表单验证
- 集成Session管理

Closes #123
\end{lstlisting}

\subsection{分支策略}

\begin{table}[H]
\centering
\caption{常用分支策略}
\begin{tabular}{llp{6cm}}
\toprule
\textbf{分支} & \textbf{用途} & \textbf{说明} \\
\midrule
main/master & 生产环境 & 始终保持稳定可发布 \\
develop & 开发环境 & 集成最新开发功能 \\
feature/* & 功能开发 & 从develop创建,完成后合并回develop \\
release/* & 发布准备 & 从develop创建,完成后合并到main和develop \\
hotfix/* & 紧急修复 & 从main创建,完成后合并到main和develop \\
\bottomrule
\end{tabular}
\end{table}

\subsection{.gitignore最佳实践}

\begin{itemize}[leftmargin=2em]
    \item 忽略编译产物(.pyc, .class, .o)
    \item 忽略依赖目录(node\_modules, venv)
    \item 忽略IDE配置(.idea, .vscode)
    \item 忽略系统文件(.DS\_Store, Thumbs.db)
    \item 忽略敏感信息(.env, secrets.json)
    \item 忽略日志文件(*.log)
    \item 不要忽略锁文件(package-lock.json, Pipfile.lock)
\end{itemize}

% ============================================================
\section{常见问题解决}
% ============================================================

\subsection{撤销已推送的提交}

\begin{lstlisting}[caption=撤销已推送的提交]
# 方法1:创建新提交来撤销(推荐)
git revert commit_hash
git push origin main

# 方法2:强制重写历史(危险)
git reset --hard HEAD^
git push -f origin main
\end{lstlisting}

\subsection{恢复删除的分支}

\begin{lstlisting}[caption=恢复删除的分支]
# 查看操作日志
git reflog

# 找到删除前的commit hash,创建新分支
git checkout -b recovered-branch commit_hash
\end{lstlisting}

\subsection{修改已推送的提交作者}

\begin{lstlisting}[caption=修改提交作者]
# 修改最近一次提交的作者
git commit --amend --author="New Name <new@email.com>"
git push -f origin main
\end{lstlisting}

\subsection{大文件处理}

\begin{lstlisting}[caption=大文件处理]
# 安装Git LFS
git lfs install

# 跟踪大文件
git lfs track "*.psd"
git lfs track "*.zip"

# 查看跟踪的文件类型
git lfs track

# 提交.gitattributes
git add .gitattributes
git commit -m "Add Git LFS tracking"
\end{lstlisting}

% ============================================================
\section{实战:智能图书管理系统提交流程}
% ============================================================

以下是本项目完整的Git操作流程:

\begin{lstlisting}[caption=完整提交流程]
# 1. 进入项目目录
cd /root/website_homework/library_system

# 2. 初始化Git仓库
git init

# 3. 创建.gitignore
cat > .gitignore << 'EOF'
__pycache__/
*.py[cod]
venv/
db.sqlite3
.env
cookies.txt
.claude/
EOF

# 4. 处理敏感信息(修改settings.py使用环境变量)

# 5. 创建.env.example
cat > .env.example << 'EOF'
AI_API_KEY=your_api_key
EMAIL_HOST_PASSWORD=your_password
EOF

# 6. 添加远程仓库
git remote add origin git@github.com:wkylnj/Library-System.git

# 7. 重命名分支
git branch -m main

# 8. 添加所有文件
git add -A

# 9. 查看状态
git status

# 10. 提交
git commit -m "Initial commit: 智能图书管理系统

功能特性:
- AI智能对话助手 (通义千问)
- AI个性化图书推荐
- AI图书封面生成 (通义万相)
- 图书副本编号管理
- 图书预约与排队系统
- 多级到期提醒
- 邮箱验证码注册
- 响应式Bootstrap 5界面"

# 11. 推送(如果远程为空)
git push -u origin main

# 12. 如果远程已有内容,强制推送
git push -f origin main
\end{lstlisting}

% ============================================================
\section{总结}
% ============================================================

\subsection{Git核心命令速查}

\begin{table}[H]
\centering
\caption{Git核心命令}
\begin{tabular}{ll}
\toprule
\textbf{命令} & \textbf{功能} \\
\midrule
git init & 初始化仓库 \\
git clone & 克隆仓库 \\
git add & 添加到暂存区 \\
git commit & 提交更改 \\
git push & 推送到远程 \\
git pull & 拉取并合并 \\
git fetch & 获取远程更新 \\
git branch & 分支管理 \\
git checkout & 切换分支/恢复文件 \\
git merge & 合并分支 \\
git rebase & 变基 \\
git stash & 暂存工作 \\
git log & 查看历史 \\
git diff & 查看差异 \\
git reset & 重置 \\
git revert & 撤销提交 \\
\bottomrule
\end{tabular}
\end{table}

\subsection{学习资源}

\begin{itemize}[leftmargin=2em]
    \item 官方文档:\url{https://git-scm.com/doc}
    \item Pro Git电子书:\url{https://git-scm.com/book/zh/v2}
    \item GitHub文档:\url{https://docs.github.com/zh}
    \item 交互式学习:\url{https://learngitbranching.js.org}
\end{itemize}

\end{document}
