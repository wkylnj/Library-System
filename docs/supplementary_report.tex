\documentclass[12pt,a4paper]{article}
\usepackage[UTF8]{ctex}
\usepackage{geometry}
\usepackage{graphicx}
\usepackage{booktabs}
\usepackage{listings}
\usepackage{xcolor}
\usepackage{hyperref}
\usepackage{enumitem}
\usepackage{fontspec}
\usepackage{titlesec}
\usepackage{fancyhdr}
\usepackage{float}
\usepackage{longtable}
\usepackage{multirow}
\usepackage{array}

% 页面设置
\geometry{left=2.5cm,right=2.5cm,top=2.5cm,bottom=2.5cm}

% 代码样式
\lstset{
    basicstyle=\ttfamily\small,
    keywordstyle=\color{blue},
    commentstyle=\color{gray},
    stringstyle=\color{red},
    backgroundcolor=\color{gray!10},
    frame=single,
    breaklines=true,
    showstringspaces=false,
    numbers=left,
    numberstyle=\tiny\color{gray}
}

% 页眉页脚
\pagestyle{fancy}
\fancyhf{}
\fancyhead[L]{智能图书管理系统}
\fancyhead[R]{后续优化更新报告}
\fancyfoot[C]{\thepage}

% 标题格式
\titleformat{\section}{\Large\bfseries}{\thesection}{1em}{}
\titleformat{\subsection}{\large\bfseries}{\thesubsection}{1em}{}
\titleformat{\subsubsection}{\normalsize\bfseries}{\thesubsubsection}{1em}{}

\begin{document}

% ============================================================
% 封面
% ============================================================
\begin{titlepage}
    \centering
    \vspace*{1.5cm}

    {\Huge\bfseries 智能图书管理系统\\[0.8cm]}
    {\LARGE\bfseries 后续优化更新报告\\[2cm]}

    \vfill

    {\Large
    \begin{tabular}{rl}
        \textbf{项目名称:} & 基于Django的智能图书管理系统 \\[0.4cm]
        \textbf{开发框架:} & Django 6.0 + Bootstrap 5 \\[0.4cm]
        \textbf{AI技术:} & 阿里云DashScope(通义系列) \\[0.4cm]
        \textbf{数据库:} & SQLite3 \\[0.4cm]
        \textbf{部署方式:} & Gunicorn + Nginx \\[0.4cm]
        \textbf{设计者:} & 王楷宇 \\[0.4cm]
        \textbf{项目仓库:} & https://github.com/wkylnj/Library-System \\[0.4cm]
    \end{tabular}
    }

    \vfill
    {\large \today}
\end{titlepage}

% 目录
\tableofcontents
\newpage

% ============================================================
\section{本次优化更新内容}
% ============================================================

本次更新在原有系统基础上进行了多项重要功能的优化与新增,主要包括图书副本编号系统、AI智能封面生成、图书预约功能以及到期提醒系统。以下对各项更新进行详细说明。

\subsection{图书副本编号系统}

\subsubsection{功能背景}

在实际图书馆场景中,同一种图书往往有多个副本。为了精确追踪每一本实体书的借阅状态,本次更新引入了图书副本编号系统。

\subsubsection{功能特性}

\begin{itemize}[leftmargin=2em]
    \item \textbf{唯一编号生成}:每个副本自动生成唯一编号,格式为"图书ID-序号"(如:0001-001、0001-002)
    \item \textbf{多状态管理}:支持5种副本状态
    \begin{itemize}
        \item 可借阅(available):副本在馆可供借阅
        \item 已借出(borrowed):副本已被用户借阅
        \item 已预约(reserved):副本已被预约锁定
        \item 维护中(maintenance):副本正在修复或整理
        \item 已丢失(lost):副本丢失或损坏报废
    \end{itemize}
    \item \textbf{品相记录}:记录每本副本的品相状态(良好、一般、较差等)
    \item \textbf{独立追踪}:借阅记录关联到具体副本,可追溯每本书的借阅历史
    \item \textbf{批量管理}:管理员可批量添加副本,支持一次添加1-50本
    \item \textbf{可视化管理}:在图书详情页直观展示所有副本状态
\end{itemize}

\subsubsection{数据模型设计}

\begin{lstlisting}[language=Python, caption=BookCopy模型定义]
class BookCopy(models.Model):
    """图书副本模型 - 每本实体书"""
    STATUS_CHOICES = [
        ('available', '可借阅'),
        ('borrowed', '已借出'),
        ('reserved', '已预约'),
        ('maintenance', '维护中'),
        ('lost', '已丢失'),
    ]

    book = models.ForeignKey(Book, on_delete=models.CASCADE,
                            related_name='copies', verbose_name='图书')
    copy_number = models.CharField(max_length=50, verbose_name='副本编号')
    status = models.CharField(max_length=20, choices=STATUS_CHOICES,
                             default='available', verbose_name='状态')
    condition = models.CharField(max_length=50, default='良好',
                                verbose_name='品相')
    notes = models.TextField(blank=True, verbose_name='备注')
    created_at = models.DateTimeField(auto_now_add=True, verbose_name='入库时间')

    class Meta:
        verbose_name = '图书副本'
        unique_together = ['book', 'copy_number']

    def save(self, *args, **kwargs):
        if not self.copy_number:
            # 自动生成副本编号
            existing_count = BookCopy.objects.filter(book=self.book).count()
            self.copy_number = f'{self.book.id:04d}-{existing_count + 1:03d}'
        super().save(*args, **kwargs)
\end{lstlisting}

\subsubsection{库存计算逻辑}

\begin{lstlisting}[language=Python, caption=动态库存计算]
class Book(models.Model):
    @property
    def total_copies(self):
        """总副本数"""
        return self.copies.count()

    @property
    def available_copies(self):
        """可借副本数"""
        return self.copies.filter(status='available').count()

    @property
    def borrowed_copies(self):
        """已借出副本数"""
        return self.copies.filter(status='borrowed').count()

    def is_available(self):
        """是否有可借副本"""
        return self.available_copies > 0

    def get_available_copy(self):
        """获取一本可借的副本"""
        return self.copies.filter(status='available').first()
\end{lstlisting}

% ------------------------------------------
\subsection{AI智能封面生成}
% ------------------------------------------

\subsubsection{功能背景}

传统图书管理系统的图书展示往往缺乏视觉吸引力。本次更新引入AI图像生成技术,为每本图书自动生成艺术风格的封面图片,大幅提升系统的视觉体验。

\subsubsection{技术方案}

\begin{itemize}[leftmargin=2em]
    \item \textbf{AI模型}:阿里云通义万相(wanx-v1)文生图大模型
    \item \textbf{调用方式}:异步任务模式,通过DashScope API提交生成任务
    \item \textbf{图片规格}:720×1280像素(竖版封面,适合移动端展示)
    \item \textbf{存储方式}:本地文件存储,路径为 media/covers/
    \item \textbf{文件命名}:{图书ID}\_{ISBN}.png
\end{itemize}

\subsubsection{智能提示词生成}

系统根据图书的标题、作者、分类信息自动生成英文提示词,并根据不同分类匹配相应的艺术风格:

\begin{lstlisting}[language=Python, caption=封面提示词生成逻辑]
def generate_cover_prompt(self, book):
    """根据图书信息生成封面提示词"""
    category_style = {
        '计算机': 'modern tech style, circuit patterns, blue and white colors,
                  digital aesthetics, futuristic design',
        '文学': 'elegant literary style, classic design, warm colors,
                vintage paper texture, artistic typography',
        '历史': 'vintage historical style, ancient textures, sepia tones,
                old manuscript feel, classical elements',
        '哲学': 'philosophical abstract style, deep colors, minimalist,
                thought-provoking imagery, symbolic elements',
        '心理学': 'mind and brain imagery, soft colors, professional,
                  neural patterns, cognitive aesthetics',
        '经济': 'business professional style, charts and graphs,
                gold and navy colors, corporate elegance',
        '科学': 'scientific illustration style, cosmos and atoms,
                vibrant colors, discovery theme',
        '艺术': 'artistic creative style, colorful palette,
                expressive brushstrokes, gallery quality',
    }

    cat_name = book.category.name if book.category else '文学'
    style = category_style.get(cat_name, 'elegant book cover design')

    prompt = f"""Book cover design for '{book.title}' by {book.author},
                {style}, high quality, professional publishing standard,
                no text on cover, artistic composition"""
    return prompt
\end{lstlisting}

\subsubsection{异步生成流程}

\begin{enumerate}
    \item 提交生成任务到DashScope API
    \item 获取任务ID,进入轮询等待
    \item 每3秒查询一次任务状态
    \item 任务成功后下载生成的图片
    \item 保存到本地并更新数据库记录
\end{enumerate}

\begin{lstlisting}[language=Python, caption=封面生成核心逻辑]
def generate_cover_with_dashscope(self, book):
    """使用DashScope API生成封面"""
    url = 'https://dashscope.aliyuncs.com/api/v1/services/aigc/
           text2image/image-synthesis'

    headers = {
        'Authorization': f'Bearer {api_key}',
        'Content-Type': 'application/json',
        'X-DashScope-Async': 'enable'  # 异步模式
    }

    data = {
        'model': 'wanx-v1',
        'input': {'prompt': self.generate_cover_prompt(book)},
        'parameters': {
            'size': '720*1280',
            'n': 1,
            'style': '<auto>'
        }
    }

    # 提交任务
    response = requests.post(url, headers=headers, json=data)
    task_id = response.json()['output']['task_id']

    # 轮询获取结果
    for _ in range(30):
        time.sleep(3)
        result = self.check_task_status(task_id)
        if result['status'] == 'SUCCEEDED':
            image_url = result['results'][0]['url']
            return self.download_image(image_url)

    return None
\end{lstlisting}

\subsubsection{管理命令}

\begin{lstlisting}[language=bash, caption=封面生成命令]
# 为所有没有封面的图书生成封面
python manage.py generate_covers

# 强制重新生成所有封面
python manage.py generate_covers --force
\end{lstlisting}

% ------------------------------------------
\subsection{图书预约功能}
% ------------------------------------------

\subsubsection{功能背景}

当热门图书的所有副本都被借出时,用户往往需要等待。预约功能允许用户在图书无库存时进行预约,系统会在有书可借时自动通知用户。

\subsubsection{功能特性}

\begin{itemize}[leftmargin=2em]
    \item \textbf{智能预约}:仅当图书无可借副本时显示预约按钮
    \item \textbf{队列管理}:按预约时间先后排序,显示用户在队列中的位置
    \item \textbf{自动通知}:图书归还时自动发送邮件通知第一个预约者
    \item \textbf{状态流转}:等待中 → 已通知 → 已完成/已过期
    \item \textbf{过期机制}:通知后3天未借阅自动过期,机会顺延给下一位
    \item \textbf{取消预约}:用户可随时取消等待中的预约
    \item \textbf{预约历史}:保留完整的预约记录供查询
\end{itemize}

\subsubsection{预约状态说明}

\begin{table}[H]
\centering
\caption{预约状态流转}
\begin{tabular}{lll}
\toprule
\textbf{状态} & \textbf{含义} & \textbf{后续操作} \\
\midrule
waiting(等待中) & 正在排队等待 & 可取消,等待通知 \\
notified(已通知) & 有书可借,已发邮件 & 3天内前往借阅 \\
fulfilled(已完成) & 用户已成功借阅 & 预约流程结束 \\
cancelled(已取消) & 用户主动取消 & 预约流程结束 \\
expired(已过期) & 通知后3天未借阅 & 机会顺延下一位 \\
\bottomrule
\end{tabular}
\end{table}

\subsubsection{数据模型设计}

\begin{lstlisting}[language=Python, caption=Reservation模型定义]
class Reservation(models.Model):
    """预约记录"""
    STATUS_CHOICES = [
        ('waiting', '等待中'),
        ('notified', '已通知'),
        ('fulfilled', '已完成'),
        ('cancelled', '已取消'),
        ('expired', '已过期'),
    ]

    user = models.ForeignKey(User, on_delete=models.CASCADE,
                            related_name='reservations', verbose_name='预约人')
    book = models.ForeignKey(Book, on_delete=models.CASCADE,
                            related_name='reservations', verbose_name='图书')
    created_at = models.DateTimeField(auto_now_add=True, verbose_name='预约时间')
    notified_at = models.DateTimeField(null=True, verbose_name='通知时间')
    status = models.CharField(max_length=20, choices=STATUS_CHOICES,
                             default='waiting', verbose_name='状态')

    @property
    def queue_position(self):
        """获取在等待队列中的位置"""
        if self.status != 'waiting':
            return None
        return Reservation.objects.filter(
            book=self.book,
            status='waiting',
            created_at__lt=self.created_at
        ).count() + 1
\end{lstlisting}

\subsubsection{自动通知逻辑}

\begin{lstlisting}[language=Python, caption=归还时自动通知预约者]
def notify_reservation(book):
    """通知该书第一个等待中的预约者"""
    reservation = Reservation.objects.filter(
        book=book,
        status='waiting'
    ).order_by('created_at').first()

    if not reservation:
        return False

    # 发送邮件通知
    subject = f'好消息!您预约的《{book.title}》现在可以借阅了'
    message = f'''
    尊敬的 {reservation.user.username}:

    您好!您预约的图书《{book.title}》现在有可借副本了。

    请在收到此邮件后3天内前往图书馆借阅,逾期预约将自动取消。

    图书信息:
    - 书名:{book.title}
    - 作者:{book.author}
    - ISBN:{book.isbn}

    祝您阅读愉快!
    图书管理系统
    '''

    send_mail(
        subject=subject,
        message=message,
        from_email=settings.DEFAULT_FROM_EMAIL,
        recipient_list=[reservation.user.email]
    )

    # 更新预约状态
    reservation.status = 'notified'
    reservation.notified_at = timezone.now()
    reservation.save()

    return True
\end{lstlisting}

% ------------------------------------------
\subsection{到期提醒系统}
% ------------------------------------------

\subsubsection{功能背景}

图书逾期是图书馆管理中的常见问题。本次更新引入了完善的到期提醒系统,通过邮件主动提醒用户,有效减少逾期情况的发生。

\subsubsection{提醒策略}

\begin{table}[H]
\centering
\caption{多级到期提醒策略}
\begin{tabular}{llp{7cm}}
\toprule
\textbf{提醒类型} & \textbf{触发时机} & \textbf{邮件内容要点} \\
\midrule
3天提醒 & 距到期日还有3天 & 温馨提示,告知到期日期,建议续借或归还 \\
1天提醒 & 距到期日还有1天 & 紧急提醒,强调明天到期,请尽快处理 \\
逾期提醒 & 已超过到期日 & 催还通知,说明逾期天数,提醒可能的后果 \\
\bottomrule
\end{tabular}
\end{table}

\subsubsection{防重复发送机制}

系统为每条借阅记录维护三个标志位,确保每种提醒只发送一次:

\begin{lstlisting}[language=Python, caption=BorrowRecord提醒标志]
class BorrowRecord(models.Model):
    # ... 其他字段 ...

    # 邮件提醒标记
    reminder_3days_sent = models.BooleanField(default=False,
                                              verbose_name='3天提醒已发送')
    reminder_1day_sent = models.BooleanField(default=False,
                                             verbose_name='1天提醒已发送')
    overdue_reminder_sent = models.BooleanField(default=False,
                                                verbose_name='逾期提醒已发送')
\end{lstlisting}

\subsubsection{提醒任务实现}

\begin{lstlisting}[language=Python, caption=到期提醒核心逻辑]
def check_due_reminders():
    """检查并发送到期提醒"""
    now = timezone.now()
    results = {'3_days': 0, '1_day': 0, 'overdue': 0}

    # 获取所有借阅中的记录
    active_borrows = BorrowRecord.objects.filter(status='borrowed')

    for record in active_borrows:
        days_remaining = (record.due_date - now).days

        # 3天提醒
        if days_remaining == 3 and not record.reminder_3days_sent:
            send_reminder_email(record, '3_days')
            record.reminder_3days_sent = True
            record.save()
            results['3_days'] += 1

        # 1天提醒
        elif days_remaining == 1 and not record.reminder_1day_sent:
            send_reminder_email(record, '1_day')
            record.reminder_1day_sent = True
            record.save()
            results['1_day'] += 1

        # 逾期提醒
        elif days_remaining < 0 and not record.overdue_reminder_sent:
            send_reminder_email(record, 'overdue')
            record.overdue_reminder_sent = True
            record.status = 'overdue'
            record.save()
            results['overdue'] += 1

    return results
\end{lstlisting}

\subsubsection{定时任务配置}

系统提供了Django管理命令,可配合cron实现定时执行:

\begin{lstlisting}[language=bash, caption=定时任务配置]
# 管理命令
python manage.py run_tasks

# Cron配置(每天上午9点执行)
0 9 * * * cd /root/website_homework/library_system && \
    /usr/bin/python manage.py run_tasks >> /var/log/library_tasks.log 2>&1
\end{lstlisting}

% ============================================================
\section{技术栈与开发流程}
% ============================================================

\subsection{技术栈详解}

\subsubsection{后端技术}

\begin{table}[H]
\centering
\caption{后端技术栈}
\begin{tabular}{llp{6cm}}
\toprule
\textbf{技术} & \textbf{版本} & \textbf{用途说明} \\
\midrule
Python & 3.12 & 编程语言,简洁高效 \\
Django & 6.0 & Web框架,提供ORM、模板、路由等 \\
SQLite & 3 & 轻量级数据库,无需单独部署 \\
Gunicorn & 最新 & WSGI服务器,处理并发请求 \\
Pillow & 最新 & 图像处理库,支持封面存储 \\
Requests & 最新 & HTTP客户端,调用AI API \\
\bottomrule
\end{tabular}
\end{table}

\subsubsection{前端技术}

\begin{table}[H]
\centering
\caption{前端技术栈}
\begin{tabular}{llp{6cm}}
\toprule
\textbf{技术} & \textbf{版本} & \textbf{用途说明} \\
\midrule
Bootstrap & 5.3 & CSS框架,响应式布局 \\
Bootstrap Icons & 最新 & 图标库,美化界面 \\
JavaScript & ES6+ & 前端交互,AJAX请求 \\
Django Template & - & 服务端模板渲染 \\
\bottomrule
\end{tabular}
\end{table}

\subsubsection{AI服务}

\begin{table}[H]
\centering
\caption{AI服务配置}
\begin{tabular}{llp{6cm}}
\toprule
\textbf{服务} & \textbf{模型} & \textbf{用途} \\
\midrule
通义千问 & qwen-turbo & AI对话、智能推荐 \\
通义万相 & wanx-v1 & 图书封面生成 \\
\bottomrule
\end{tabular}
\end{table}

\subsubsection{部署架构}

\begin{table}[H]
\centering
\caption{部署技术}
\begin{tabular}{llp{6cm}}
\toprule
\textbf{组件} & \textbf{技术} & \textbf{说明} \\
\midrule
Web服务器 & Nginx & 反向代理,静态文件服务 \\
应用服务器 & Gunicorn & WSGI服务,Unix Socket通信 \\
进程管理 & Systemd & 服务管理,开机自启 \\
\bottomrule
\end{tabular}
\end{table}

\subsection{项目结构}

\begin{lstlisting}[caption=项目目录结构]
library_system/
├── config/                 # 项目配置
│   ├── settings.py        # Django配置
│   ├── urls.py            # 主路由
│   └── wsgi.py            # WSGI入口
├── books/                  # 图书应用
│   ├── models.py          # 数据模型
│   ├── views.py           # 视图函数
│   ├── urls.py            # 应用路由
│   ├── forms.py           # 表单定义
│   ├── admin.py           # 后台管理
│   ├── ai_chat.py         # AI对话服务
│   ├── ai_recommend.py    # AI推荐服务
│   ├── tasks.py           # 定时任务
│   └── management/        # 管理命令
│       └── commands/
│           ├── run_tasks.py
│           └── generate_covers.py
├── users/                  # 用户应用
│   ├── models.py          # 用户模型
│   ├── views.py           # 认证视图
│   └── utils.py           # 邮件工具
├── templates/              # 模板文件
│   ├── base.html          # 基础模板
│   ├── books/             # 图书相关模板
│   └── users/             # 用户相关模板
├── static/                 # 静态文件
├── media/                  # 上传文件
│   └── covers/            # AI生成的封面
└── manage.py              # Django管理脚本
\end{lstlisting}

\subsection{开发流程}

\subsubsection{需求分析阶段}

\begin{enumerate}
    \item 调研图书馆管理系统的核心需求
    \item 分析用户角色:普通用户、管理员
    \item 确定功能模块:图书管理、借阅管理、用户管理、AI功能
    \item 设计数据库ER图和系统架构图
\end{enumerate}

\subsubsection{设计阶段}

\begin{enumerate}
    \item 设计数据库模型:User、Book、BookCopy、BorrowRecord、Reservation等
    \item 设计URL路由结构
    \item 设计页面原型和交互流程
    \item 确定AI服务接入方案
\end{enumerate}

\subsubsection{开发阶段}

\begin{enumerate}
    \item 搭建Django项目框架
    \item 实现数据模型和数据库迁移
    \item 开发视图和模板
    \item 集成AI服务(对话、推荐、封面生成)
    \item 实现邮件通知功能
    \item 开发管理员后台
\end{enumerate}

\subsubsection{测试部署阶段}

\begin{enumerate}
    \item 功能测试:验证各模块功能正确性
    \item 配置Gunicorn和Nginx
    \item 部署到云服务器
    \item 配置定时任务
\end{enumerate}

% ============================================================
\section{系统功能与特色总览}
% ============================================================

\subsection{AI智能对话助手}

\subsubsection{功能描述}

AI智能对话助手是本系统的核心亮点功能之一,基于阿里云通义千问大语言模型,为用户提供自然、流畅的对话交互体验。

\subsubsection{核心能力}

\begin{itemize}[leftmargin=2em]
    \item \textbf{自然语言理解}:准确理解用户的自然语言问题
    \item \textbf{多轮对话}:支持上下文连续对话,理解对话历史
    \item \textbf{图书馆知识}:了解馆藏图书信息,可回答相关问题
    \item \textbf{阅读建议}:根据用户需求推荐合适的书籍
    \item \textbf{使用指导}:解答系统使用相关问题
\end{itemize}

\subsubsection{技术实现}

\begin{lstlisting}[language=Python, caption=AI对话服务实现]
class AIChatService:
    def __init__(self):
        self.api_key = settings.AI_API_KEY
        self.api_url = settings.AI_API_URL
        self.model = settings.AI_MODEL

    def build_system_prompt(self):
        """构建系统提示词"""
        books_info = self.get_books_summary()
        return f"""你是图书馆的AI智能助手,名叫"小图"。
        你的设计者是王楷宇,李念骏是她的女朋友。

        你的职责是:
        1. 回答用户关于图书馆的问题
        2. 推荐适合用户的书籍
        3. 介绍图书内容和作者信息
        4. 帮助用户使用图书馆系统

        当前图书馆藏书概况:{books_info}

        请用友好、专业的语气回答用户问题。"""

    def chat(self, user_message, conversation_history):
        """处理用户消息"""
        messages = [
            {'role': 'system', 'content': self.build_system_prompt()}
        ]

        # 添加历史对话
        for msg in conversation_history[-10:]:  # 保留最近10轮
            messages.append(msg)

        messages.append({'role': 'user', 'content': user_message})

        # 调用API
        response = requests.post(
            self.api_url,
            headers={'Authorization': f'Bearer {self.api_key}'},
            json={
                'model': self.model,
                'messages': messages,
                'temperature': 0.7,
                'max_tokens': 1000
            }
        )

        result = response.json()
        return {
            'success': True,
            'message': result['choices'][0]['message']['content']
        }
\end{lstlisting}

\subsubsection{用户界面}

\begin{itemize}[leftmargin=2em]
    \item 聊天气泡式对话界面
    \item 实时显示AI回复
    \item 快捷问题按钮
    \item 对话历史记录
    \item 支持发送图书相关问题
\end{itemize}

% ------------------------------------------
\subsection{AI个性化推荐算法}
% ------------------------------------------

\subsubsection{功能描述}

基于用户借阅历史和阅读偏好,利用AI大模型生成个性化的图书推荐,每条推荐都附带推荐理由。

\subsubsection{推荐算法流程}

\begin{enumerate}
    \item \textbf{用户画像构建}
    \begin{itemize}
        \item 分析用户借阅历史
        \item 统计偏好分类(按借阅次数排序)
        \item 提取喜爱的作者
        \item 计算总借阅量
    \end{itemize}

    \item \textbf{候选书籍筛选}
    \begin{itemize}
        \item 获取所有有库存的图书
        \item 排除用户已借阅过的书籍
        \item 准备图书的标题、作者、分类、简介信息
    \end{itemize}

    \item \textbf{AI推荐生成}
    \begin{itemize}
        \item 构建包含用户画像和候选书籍的提示词
        \item 调用AI模型生成推荐结果
        \item 解析返回的JSON格式推荐
    \end{itemize}

    \item \textbf{结果呈现}
    \begin{itemize}
        \item 展示推荐书籍列表
        \item 显示每本书的推荐理由
        \item 提供快速借阅入口
    \end{itemize}
\end{enumerate}

\subsubsection{用户画像构建}

\begin{lstlisting}[language=Python, caption=用户阅读画像分析]
def get_user_reading_profile(self, user):
    """获取用户阅读画像"""
    borrow_records = BorrowRecord.objects.filter(user=user)\
                                         .select_related('book', 'book__category')

    if not borrow_records.exists():
        return None

    # 统计分类偏好
    category_counts = {}
    authors = []
    books_borrowed = []

    for record in borrow_records:
        book = record.book
        books_borrowed.append(book.title)

        if book.category:
            cat_name = book.category.name
            category_counts[cat_name] = category_counts.get(cat_name, 0) + 1

        if book.author:
            authors.append(book.author)

    # 按借阅次数排序,取前3个分类
    favorite_categories = sorted(
        category_counts.items(),
        key=lambda x: x[1],
        reverse=True
    )[:3]

    return {
        'books_borrowed': books_borrowed,
        'favorite_categories': [cat[0] for cat in favorite_categories],
        'favorite_authors': list(set(authors))[:5],
        'total_borrowed': len(books_borrowed),
    }
\end{lstlisting}

\subsubsection{备选推荐机制}

当AI服务不可用时,系统自动降级为基于规则的推荐:

\begin{lstlisting}[language=Python, caption=规则推荐备选方案]
def get_rule_based_recommendations(self, user_profile, available_books):
    """基于规则的推荐(AI不可用时的备选方案)"""
    recommendations = []
    borrowed_titles = set(user_profile['books_borrowed']) if user_profile else set()

    # 1. 推荐用户喜欢的分类中的书
    if user_profile and user_profile['favorite_categories']:
        for cat in user_profile['favorite_categories']:
            books = Book.objects.annotate(
                avail_copies=Count('copies', filter=Q(copies__status='available'))
            ).filter(category__name=cat, avail_copies__gt=0)\
             .exclude(title__in=borrowed_titles)[:2]

            for book in books:
                recommendations.append({
                    'book': book,
                    'reason': f'根据您对"{cat}"类书籍的喜好推荐'
                })

    # 2. 补充热门书籍
    if len(recommendations) < 5:
        popular_books = Book.objects.annotate(
            avail_copies=Count('copies', filter=Q(copies__status='available')),
            borrow_count=Count('borrow_records')
        ).filter(avail_copies__gt=0)\
         .exclude(title__in=borrowed_titles)\
         .order_by('-borrow_count')[:5 - len(recommendations)]

        for book in popular_books:
            recommendations.append({
                'book': book,
                'reason': '图书馆热门藏书推荐'
            })

    return recommendations[:5]
\end{lstlisting}

\subsubsection{相似书籍推荐}

在图书详情页,系统自动推荐与当前图书相似的书籍:

\begin{lstlisting}[language=Python, caption=相似书籍推荐]
def get_similar_books(self, book):
    """获取相似书籍推荐"""
    similar = []

    # 同分类的书
    if book.category:
        same_category = Book.objects.annotate(
            avail_copies=Count('copies', filter=Q(copies__status='available'))
        ).filter(category=book.category, avail_copies__gt=0)\
         .exclude(id=book.id)[:3]

        for b in same_category:
            similar.append({
                'book': b,
                'reason': f'同属"{book.category.name}"分类'
            })

    # 同作者的书
    same_author = Book.objects.annotate(
        avail_copies=Count('copies', filter=Q(copies__status='available'))
    ).filter(author=book.author, avail_copies__gt=0)\
     .exclude(id=book.id)[:2]

    for b in same_author:
        similar.append({
            'book': b,
            'reason': f'同为{book.author}的作品'
        })

    return similar[:5]
\end{lstlisting}

% ------------------------------------------
\subsection{AI图书封面生成}
% ------------------------------------------

(详见第1.2节)

\subsubsection{功能亮点}

\begin{itemize}[leftmargin=2em]
    \item \textbf{全自动化}:无需人工干预,系统自动为图书生成封面
    \item \textbf{风格多样}:根据图书分类智能匹配艺术风格
    \item \textbf{高质量输出}:720×1280高清图片,适合各种展示场景
    \item \textbf{批量处理}:支持一键为所有图书生成封面
    \item \textbf{视觉提升}:大幅提升图书列表的视觉吸引力
\end{itemize}

% ------------------------------------------
\subsection{图书副本序号管理}
% ------------------------------------------

(详见第1.1节)

\subsubsection{管理功能}

\begin{itemize}[leftmargin=2em]
    \item 查看图书所有副本列表
    \item 查看每个副本的状态和品相
    \item 修改副本状态(可借阅/维护中/已丢失)
    \item 批量添加新副本
    \item 追踪副本借阅历史
\end{itemize}

% ------------------------------------------
\subsection{图书借阅管理}
% ------------------------------------------

\subsubsection{借阅功能}

\begin{itemize}[leftmargin=2em]
    \item \textbf{一键借阅}:用户点击按钮即可完成借阅
    \item \textbf{自动分配}:系统自动分配一本可用副本
    \item \textbf{借阅期限}:默认30天借阅期
    \item \textbf{借阅限制}:同一本书只能借阅一次
    \item \textbf{副本关联}:借阅记录关联到具体副本编号
\end{itemize}

\subsubsection{归还功能}

\begin{itemize}[leftmargin=2em]
    \item \textbf{快速归还}:在个人中心一键归还
    \item \textbf{状态更新}:自动更新副本状态为可借阅
    \item \textbf{预约通知}:归还时自动通知预约者
    \item \textbf{记录保存}:保留完整借阅历史
\end{itemize}

\subsubsection{借阅流程}

\begin{lstlisting}[language=Python, caption=借阅流程实现]
@login_required
def borrow_book(request, pk):
    """借阅图书"""
    book = get_object_or_404(Book, pk=pk)

    # 检查是否已借阅
    existing_borrow = BorrowRecord.objects.filter(
        user=request.user, book=book, status='borrowed'
    ).exists()

    if existing_borrow:
        messages.warning(request, '您已经借阅了这本书。')
        return redirect('books:detail', pk=pk)

    # 获取可借的副本
    available_copy = book.get_available_copy()

    if not available_copy:
        messages.error(request, '该图书暂无可借阅的副本。')
        return redirect('books:detail', pk=pk)

    # 更新副本状态
    available_copy.status = 'borrowed'
    available_copy.save()

    # 创建借阅记录
    BorrowRecord.objects.create(
        user=request.user,
        book=book,
        book_copy=available_copy
    )

    # 如果用户有预约,标记为已完成
    Reservation.objects.filter(
        user=request.user, book=book, status__in=['waiting', 'notified']
    ).update(status='fulfilled')

    messages.success(request,
        f'成功借阅《{book.title}》[{available_copy.copy_number}]')
    return redirect('users:profile')
\end{lstlisting}

% ------------------------------------------
\subsection{管理员仪表盘}
% ------------------------------------------

\subsubsection{功能描述}

管理员仪表盘提供系统运行状态的全面概览,帮助管理员快速了解图书馆运营情况。

\subsubsection{统计数据}

\begin{itemize}[leftmargin=2em]
    \item \textbf{图书总数}:系统中的图书种类数量
    \item \textbf{副本总数}:所有图书副本的总数
    \item \textbf{借阅中}:当前正在借阅的记录数
    \item \textbf{逾期借阅}:超过应还日期的借阅数
    \item \textbf{待处理预约}:等待中的预约数量
\end{itemize}

\subsubsection{快捷操作}

\begin{itemize}[leftmargin=2em]
    \item 添加图书
    \item 管理分类
    \item 查看所有借阅记录
    \item 查看所有预约记录
    \item 进入Django后台管理
\end{itemize}

\subsubsection{最近借阅}

实时显示最近10条借阅记录,包括:
\begin{itemize}
    \item 借阅用户
    \item 借阅图书
    \item 借阅日期
    \item 应还日期
    \item 当前状态
\end{itemize}

% ------------------------------------------
\subsection{用户注册与登录}
% ------------------------------------------

\subsubsection{邮箱验证码注册}

\begin{itemize}[leftmargin=2em]
    \item \textbf{6位数字验证码}:安全可靠,易于输入
    \item \textbf{验证码有效期}:24小时内有效
    \item \textbf{邮件发送}:使用SMTP发送验证邮件
    \item \textbf{重发机制}:支持重新发送验证码
\end{itemize}

\begin{lstlisting}[language=Python, caption=验证码生成]
def generate_verification_code():
    """生成6位数字验证码"""
    return ''.join(random.choices(string.digits, k=6))

class EmailVerificationToken(models.Model):
    user = models.ForeignKey(User, on_delete=models.CASCADE)
    token = models.CharField(max_length=100)
    code = models.CharField(max_length=6, default=generate_verification_code)
    created_at = models.DateTimeField(auto_now_add=True)

    def is_valid(self):
        # 24小时内有效
        return timezone.now() < self.created_at + timedelta(hours=24)
\end{lstlisting}

\subsubsection{用户登录}

\begin{itemize}[leftmargin=2em]
    \item 支持用户名/邮箱登录
    \item 记住登录状态
    \item 登录后跳转到原页面
    \item 密码错误提示
\end{itemize}

\subsubsection{个人中心}

\begin{itemize}[leftmargin=2em]
    \item 查看个人信息
    \item 修改个人资料
    \item 查看当前借阅
    \item 查看借阅历史
    \item 管理我的预约
\end{itemize}

% ------------------------------------------
\subsection{系统安全性}
% ------------------------------------------

\subsubsection{认证与授权}

\begin{itemize}[leftmargin=2em]
    \item \textbf{用户认证}:基于Django内置认证系统
    \item \textbf{登录保护}:敏感操作需要登录
    \item \textbf{权限控制}:区分普通用户和管理员权限
    \item \textbf{邮箱验证}:注册需验证邮箱真实性
\end{itemize}

\subsubsection{CSRF防护}

\begin{itemize}[leftmargin=2em]
    \item 所有POST请求都需要CSRF Token
    \item Django自动生成和验证Token
    \item 防止跨站请求伪造攻击
\end{itemize}

\begin{lstlisting}[language=HTML, caption=CSRF Token使用]
<form method="post">
    
    <button type="submit">提交</button>
</form>
\end{lstlisting}

\subsubsection{XSS防护}

\begin{itemize}[leftmargin=2em]
    \item Django模板自动转义HTML
    \item 用户输入内容安全显示
    \item 防止跨站脚本攻击
\end{itemize}

\subsubsection{SQL注入防护}

\begin{itemize}[leftmargin=2em]
    \item 使用Django ORM进行数据库操作
    \item 参数化查询,避免SQL注入
    \item 不使用原始SQL语句
\end{itemize}

\subsubsection{密码安全}

\begin{itemize}[leftmargin=2em]
    \item 密码使用PBKDF2算法加密存储
    \item 不存储明文密码
    \item 密码强度验证(可配置)
\end{itemize}

\subsubsection{API密钥保护}

\begin{itemize}[leftmargin=2em]
    \item AI API密钥存储在配置文件
    \item 不在前端暴露API密钥
    \item 生产环境使用环境变量
\end{itemize}

% ------------------------------------------
\subsection{其他功能特色}
% ------------------------------------------

\subsubsection{响应式设计}

\begin{itemize}[leftmargin=2em]
    \item 基于Bootstrap 5框架
    \item 适配桌面、平板、手机
    \item 移动端友好的操作界面
    \item 图片自适应缩放
\end{itemize}

\subsubsection{优雅的首页设计}

\begin{itemize}[leftmargin=2em]
    \item 全屏书架背景欢迎页
    \item 渐变遮罩层美化
    \item 滚动动画效果
    \item 快捷入口按钮
\end{itemize}

\subsubsection{图书搜索}

\begin{itemize}[leftmargin=2em]
    \item 关键词搜索(书名、作者、ISBN、出版社)
    \item 分类筛选
    \item 分页显示
    \item 快速定位
\end{itemize}

\subsubsection{分类管理}

\begin{itemize}[leftmargin=2em]
    \item 自定义图书分类
    \item 分类增删改查
    \item 分类关联图书统计
\end{itemize}

\subsubsection{图书管理}

\begin{itemize}[leftmargin=2em]
    \item 图书信息录入
    \item ISBN验证
    \item 图书编辑和删除
    \item 批量添加副本
    \item 封面上传/AI生成
\end{itemize}

\subsubsection{邮件通知系统}

\begin{itemize}[leftmargin=2em]
    \item 注册验证码邮件
    \item 预约成功通知
    \item 图书可借通知
    \item 到期提醒邮件
    \item 逾期催还邮件
\end{itemize}

% ============================================================
\section{系统功能清单}
% ============================================================

\begin{longtable}{p{2.5cm}p{5cm}p{4cm}p{2cm}}
\caption{完整功能清单} \\
\toprule
\textbf{功能模块} & \textbf{功能描述} & \textbf{技术实现} & \textbf{状态} \\
\midrule
\endfirsthead
\multicolumn{4}{c}{\tablename\ \thetable{} -- 续} \\
\toprule
\textbf{功能模块} & \textbf{功能描述} & \textbf{技术实现} & \textbf{状态} \\
\midrule
\endhead
\midrule
\multicolumn{4}{r}{续下页} \\
\endfoot
\bottomrule
\endlastfoot

\multicolumn{4}{l}{\textbf{AI智能功能}} \\
\midrule
AI对话助手 & 自然语言多轮对话 & 通义千问qwen-turbo & 已完成 \\
AI图书推荐 & 个性化推荐+推荐理由 & 用户画像+大模型 & 已完成 \\
AI封面生成 & 自动生成艺术封面 & 通义万相wanx-v1 & 已完成 \\
相似推荐 & 图书详情页相似书推荐 & 分类+作者匹配 & 已完成 \\
\midrule

\multicolumn{4}{l}{\textbf{图书管理}} \\
\midrule
图书列表 & 分页展示所有图书 & Django分页器 & 已完成 \\
图书搜索 & 关键词+分类搜索 & ORM Query & 已完成 \\
图书详情 & 展示图书完整信息 & 模板渲染 & 已完成 \\
图书添加 & 管理员添加图书 & 表单验证 & 已完成 \\
图书编辑 & 管理员编辑图书 & ModelForm & 已完成 \\
图书删除 & 管理员删除图书 & 确认删除 & 已完成 \\
封面管理 & 上传/AI生成封面 & ImageField & 已完成 \\
\midrule

\multicolumn{4}{l}{\textbf{副本管理}} \\
\midrule
副本编号 & 自动生成唯一编号 & 格式:ID-序号 & 已完成 \\
副本状态 & 5种状态管理 & 状态机 & 已完成 \\
批量添加 & 一次添加多个副本 & 循环创建 & 已完成 \\
状态修改 & 管理员修改状态 & AJAX更新 & 已完成 \\
副本列表 & 查看所有副本 & 表格展示 & 已完成 \\
\midrule

\multicolumn{4}{l}{\textbf{借阅管理}} \\
\midrule
图书借阅 & 一键借阅图书 & 自动分配副本 & 已完成 \\
图书归还 & 在线归还图书 & 状态更新 & 已完成 \\
借阅记录 & 查看借阅历史 & 分页列表 & 已完成 \\
借阅期限 & 30天借阅期 & 自动计算 & 已完成 \\
逾期标记 & 自动标记逾期 & 定时检查 & 已完成 \\
\midrule

\multicolumn{4}{l}{\textbf{预约功能}} \\
\midrule
图书预约 & 无库存时预约 & 队列管理 & 已完成 \\
队列位置 & 显示排队位置 & 实时计算 & 已完成 \\
自动通知 & 有书时邮件通知 & 事件触发 & 已完成 \\
取消预约 & 用户取消预约 & 状态更新 & 已完成 \\
预约过期 & 3天未借自动过期 & 定时任务 & 已完成 \\
预约历史 & 查看预约记录 & 分页列表 & 已完成 \\
\midrule

\multicolumn{4}{l}{\textbf{提醒系统}} \\
\midrule
3天提醒 & 到期前3天邮件提醒 & 定时任务 & 已完成 \\
1天提醒 & 到期前1天紧急提醒 & 定时任务 & 已完成 \\
逾期提醒 & 逾期后催还通知 & 定时任务 & 已完成 \\
防重发送 & 每种提醒只发一次 & 标志位 & 已完成 \\
\midrule

\multicolumn{4}{l}{\textbf{用户系统}} \\
\midrule
用户注册 & 邮箱验证码注册 & 6位验证码 & 已完成 \\
用户登录 & 用户名/邮箱登录 & Django认证 & 已完成 \\
个人中心 & 查看个人信息 & 模板展示 & 已完成 \\
资料编辑 & 修改个人资料 & 表单提交 & 已完成 \\
我的借阅 & 当前+历史借阅 & 分类展示 & 已完成 \\
我的预约 & 预约记录管理 & 状态筛选 & 已完成 \\
\midrule

\multicolumn{4}{l}{\textbf{管理后台}} \\
\midrule
仪表盘 & 统计数据概览 & 卡片展示 & 已完成 \\
借阅管理 & 所有借阅记录 & 状态筛选 & 已完成 \\
预约管理 & 所有预约记录 & 状态筛选 & 已完成 \\
分类管理 & 分类增删改查 & CRUD & 已完成 \\
Django后台 & 数据库管理 & Admin & 已完成 \\
\midrule

\multicolumn{4}{l}{\textbf{系统安全}} \\
\midrule
CSRF防护 & 防跨站请求伪造 & Token验证 & 已完成 \\
XSS防护 & 防跨站脚本攻击 & 自动转义 & 已完成 \\
SQL防护 & 防SQL注入 & ORM & 已完成 \\
密码加密 & PBKDF2加密存储 & Django Auth & 已完成 \\
权限控制 & 用户/管理员权限 & 装饰器 & 已完成 \\
\midrule

\multicolumn{4}{l}{\textbf{界面设计}} \\
\midrule
响应式布局 & 多设备适配 & Bootstrap 5 & 已完成 \\
欢迎页面 & 全屏背景欢迎页 & CSS动画 & 已完成 \\
图标美化 & 丰富的图标 & Bootstrap Icons & 已完成 \\
消息提示 & 操作反馈提示 & Django Messages & 已完成 \\

\end{longtable}

% ============================================================
\section{总结}
% ============================================================

本次后续优化更新为智能图书管理系统带来了多项重要功能的提升:

\subsection{核心更新}

\begin{enumerate}
    \item \textbf{图书副本编号系统}:实现了同一图书多副本的精细化管理,每个副本拥有唯一编号,可独立追踪借阅状态。

    \item \textbf{AI智能封面生成}:利用阿里云通义万相大模型,为图书自动生成艺术风格的封面,大幅提升系统视觉体验。

    \item \textbf{图书预约功能}:当图书无可借副本时,用户可进行预约排队,系统自动在有书时发送邮件通知。

    \item \textbf{到期提醒系统}:建立了3天、1天、逾期的多级提醒机制,有效减少图书逾期情况。
\end{enumerate}

\subsection{技术亮点}

\begin{enumerate}
    \item \textbf{AI三位一体}:集成了AI对话、AI推荐、AI封面生成三大AI功能,形成完整的智能化体验。

    \item \textbf{完善的借阅流程}:从借阅、预约、提醒到归还,形成闭环的借阅管理流程。

    \item \textbf{精细化管理}:副本级别的状态追踪,为图书馆运营提供精确数据支持。

    \item \textbf{自动化运维}:定时任务自动处理到期提醒和预约过期,减少人工干预。
\end{enumerate}

\subsection{系统特色}

\begin{enumerate}
    \item 深度融合AI技术,提供智能化服务体验
    \item 完善的邮件通知系统,保持与用户的良好沟通
    \item 响应式设计,支持多终端访问
    \item 安全可靠的用户认证和数据保护机制
    \item 直观的管理后台,便于系统运维
\end{enumerate}

\end{document}
